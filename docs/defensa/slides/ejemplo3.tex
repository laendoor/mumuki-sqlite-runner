Ejemplo de un ejercicio de tipo \texttt{CREATE} con solución vía \textit{Datasets}

\begin{columns}[t]
    \column{.5\textwidth}
    \begin{listing}[H]
        \caption{Extra (Docente, SQL)}
        \begin{minted}[fontsize=\scriptsize]{sql}
    -- NONE
        \end{minted}
    \end{listing}

    \vspace{-1em}
    \begin{listing}[H]
        \caption{Content (Alumno, SQL)}
        \begin{minted}[fontsize=\tiny]{sql}
    CREATE TABLE motores (
        id integer primary key,
        nombre varchar(200) NOT NULL
    );

    INSERT INTO motores values ('MySQL');
    INSERT INTO motores values ('PostgreSQL');
    INSERT INTO motores values ('Oracle');
    INSERT INTO motores values ('SQL Server');
    INSERT INTO motores values ('SQLite');

    SELECT * FROM motores;
        \end{minted}
    \end{listing}

    \column{.5\textwidth}
    \begin{listing}[H]
        \caption{Test (Docente, YAML)}
        \begin{minted}[fontsize=\tiny]{yaml}
solution_type: datasets
examples:
  - data: -- none
    solution_dataset: |
      id|color
      1|MySQL
      2|PostgreSQL
      3|Oracle
      4|SQL Server
      4|SQLite
        \end{minted}
    \end{listing}

\end{columns}
