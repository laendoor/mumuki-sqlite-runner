
La materia \textit{Bases de Datos} contiene mayormente
alumnos en su segundo o tercer cuatrimestre para
la Tecnicatura en Programación dictada en la Universidad
Nacional de Quilmes. Idealmente son alumnos que
vienen de cursar las materias \textit{Introducción
a la Programación} y \textit{Organización de Computadoras},
materias antagónicas desde el \textit{alto} y \textit{bajo nivel}
de los conceptos transmitidos. En Intro se aprenden conceptos
de alto nivel usando Gobstones y en Orga conceptos de nivel
de máquina utilizando QSIM\footnote{\url{http://orga.blog.unq.edu.ar/qsim/}}.
Y al mismo tiempo que Bases de Datos muchos cursan
\textit{Programación con Objetos I}.

Desde \textit{BBDD} se busca transmitir el concepto de relación
de información. Se enseñan conceptos como \textit{Modelo de
Entidad-Relacion}, \textit{Normalización} y \textit{Álgebra Relacional}.
El puente entre estos conceptos y el mundo de la programación
es el \textbf{lenguaje SQL}.

Al querer enseñar los conceptos necesarios para que el alumno
pueda manejar SQL, aparecen los problemas recurrentes asociados
a la tecnología. Los motores utilizados en la industria como
Oracle, Postgres, SQL Server, MySQL, etc., son piezas de software
tan robustas como complejas, y pueden ser una demora cuando se
quieren trasmitir otras ideas fundacionales.

Mumuki cuenta con soporte para Gobstones~\footnote{\url{https://github.com/mumuki/mumuki-gobstones-runner}}~\cite{MumukiGobstonesAloi}
y para QSIM\footnote{\url{https://github.com/mumuki/mumuki-qsim-runner}},
y ambos son utilizados en Intro y Orga respectivamente.
Además de la ventaja de la familiarización de la tecnología,
Mumuki permite incorporar los componentes necesarios para
obtener una adaptación a la necesidad puntual~\footnote{dado que
es Open Source}.


\subsection{Proyecto Mumuki}

\begin{displayquote}[\cite{PaperMumuki}]
``Mumuki es un software educativo para aprender a programar a partir de la resolución de problemas;
plantea enseñar conceptos de programación, en un proceso conducido por guías prácticas
en las que la teoría surge a medida que se avanza. Esta herramienta se presenta al
estudiante como una aplicación Web interactiva, en la que se articulan explicaciones
y ejemplos con la opción de que cada uno realice su propia solución y la plataforma
la pruebe y corrija instantáneamente, orientando acerca de los aciertos y errores.''
\end{displayquote}

Mumuki es el resultado del trabajo de muchas personas que creen
que es necesario plantear nuevas formas de enseñar a programar.
Lo que se plantea es un complemento a la clase en el aula, no
un reemplazo, pero haciendo foco en que la plataforma pueda ayudar
a resolver el mayor número de \textit{problemas técnicos}
del alumno con la tacnología y permitir al docente enfocarse
en las problemáticas inherentes a la enseñanza de los conceptos.

Es común el cursos introductorios de programación orientada a objetos que los alumnos
estén una o dos clases (y el tiempo entre semana) tratando de hacer funcionar
el entorno de Java con sus bibliotecas y el IDE, en lugar de
aprovechar ese tiempo para practicar conceptos elementales como mensajes
entre objetos, comportamiento, herencia, etc.

Al usar una plataforma como Mumuki el docente puede saltear esta problemática
y enforcarse en lo conceptual. Por supuesto que todo estudiante de programación
debe lograr aprender a instalar y configurar entornos de desarrollo, pero
de esta forma evitan muchas frustaciones inherentes a la tecnología
que terminan siendo asociadas a la programación.

\subsection{Gobstones como hilo conductor}

Si bien este trabajo no se relaciona directamente con Gobstones
\footnote{\url{http://www.gobstones.org/}},
gran parte de las ideas utilizadas tienen como uno
de los orígenes las planteadas en el libro
\enquote{Las bases conceptuales de la Programación: Una nueva forma de aprender a programar}~\cite{LibroGobstones}, del cual obtenemos su definición:

\begin{displayquote}
``Gobstones es un lenguaje conciso de sintaxis razonablemente simple,
orientado a personas que no tienen conocimientos previos en programación.
El lenguaje maneja distintos componentes propios, ideados con el fin de
aprender a resolver problemas en programación,  pero al mismo tiempo
intentando volver atractivo el aprendizaje, para lograr captar la atención
y capacidad de asombro del estudiante.''
\end{displayquote}

Gobstones busca facilitar la enseñanza de conceptos iniciales de programación
tratando de mantener distancia de las problemáticas técnicas y
la comprensión de errores complejos ante fallos de programas.
Cuenta con el entorno \textit{PyGobstones}~\footnote{\url{http://inpr.web.unq.edu.ar/instalacion-de-pygobstones/}} programado en python
pero actualmente se está desarrollando una versión web~\footnote{https://gobstones.github.io/editor-beta/} que no requiere instalación.


