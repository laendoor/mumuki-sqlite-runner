
Hay nuevas formas de aprender a programar, de enseñar a programar
y también nuevas formas de programar. No tiene que ver con gustos or criterios
sino con una realidad. La programación como motor de la computación y
las comunicaciones cambiaron paradigmáticamente la interacción de las
personas y las sociedades a nivel mundial.
Hoy existe una fuerte necesidad de instantaneidad en las respuestas.
En cierto punto se sabe que la respuesta está a la vuelta
\textit{de una googleada}, sólo es necesario indicar las palabras
de búsqueda mágicas.

Esta velocidad de la información repercute en las prácticas de
la enseñanza formal, la cual necesita moverse por terreno firme
antes que veloz. Cada momento didáctico con un grupo de alumnos
es una oportunidad única de transmisión de conocimiento; y de la
calidad del conocimiento transmitido.
Pero el conocimiento no puede ser transmitido sólo por la calidad del docente.
El alumno tiene que estar dispuesto a aceptar ese conocimiento.

Aprender programación es un desafío en sí mismo. Es necesario poder
manejar herramientas tecnológicas como lenguajes de programación,
entornos de desarrollo, compiladores, intérpretes, etc. Pero también
conceptos como abstracción, refactorización, parametrización.
Los flujos y los protocolos de comunicación, las
estructuras de datos: árboles, listas, pilas, colas, etc.
Y todo sin dejar de tener que cuenta que es necesario ser ordenado y declarativo
en el diseño de programas, pues será leído por otras personas que
tienen que poder entenderlo.

Es fundamental entonces organizar estos conceptos y presentarlos
de forma que no sea abrumadora para el alumno. Es importante poder
minimizar los problemas de una herramienta tecnológica cuando
se busca enseñar un concepto o una estructura.
Y es importante que el alumno logre sentirse atraído por ese conocimiento.

Gran parte de los profesores y alumnos están muy naturalizado con el
uso de ciertas tecnologías (computadoras, celulares, internet, etc.),
las cuales fueron construidas por docentes y profesionales de la programación.
La próxima generación ya usa activamente las herramientas que se busca
enseñarles a construir. Es posible entonces utilizar estas mismas herramientas
para enseñar a construirlas.

