Cuando se enciende la computadora, el motherboard recibe energía y se prende.
Al encenderse, el mother inicializa su firmware y trata de hacer andar el CPU.
Si todo anduvo bien el CPU comienza a andar. En un sistema multi-procesador
o multi-core se selecciona un CPU (al azar) como procesador de arranque
(bootstrap processor, o BSP) que ejecuta el BIOS (Basic Input Output System)
y el código de inicialización del kernel del sistema operativo. 


\begin{myframe}
    El BIOS es el primer programa que se ejecuta en la computadora,
    su propósito fundamental es iniciar y probar el hardware del sistema
    y cargar un gestor de arranque o un sistema operativo desde
    algún dispositivo de almacenamiento.
\end{myframe}

La mayor parte de los registros del CPU tienen valores definidos en el arranque,
incluyendo el EIP (puntero de instrucciones) que guarda la dirección de memoria
de la instrucción que se está ejecutando por la CPU. 
La CPU luego comienza a ejecutar código del BIOS, que inicializa algunos
de los componentes de hardware de la máquina. Hecho esto, la BIOS lanza
el POST (Power on self test) que chequea varios componentes de la computadora. 

\begin{myframe}
    POST es un proceso de verificación e inicialización de los componentes
    de entrada y salida en un sistema de cómputo que se encarga de configurar
    y diagnosticar el estado del hardware.
\end{myframe}

El POST involucra una serie de tests e inicializaciones, incluyendo descubrimiento
de recursos -interrupciones, rangos de memoria, puertos de entrada/salida-
para los dispositivos PCI. Los BIOS modernos que siguen las especificaciones
ACPI (Advanced Configuration and Power Interface) construyen unas tablas de
datos que describen los dispositivos en la computadora; estas tablas son usadas luego por el kernel.

Después del POST, el BIOS intenta bootear un sistema operativo, que debe estar en
alguna parte: discos rígidos, CD/DVDs, USBs, u otros dispositivos de almacenamiento.
El orden en el que el BIOS busca el dispositivo de booteo es configurable por el usuario.
Si  no hay ningun dispositivo de booteo el BIOS termina con un mensaje de error del estilo
\enquote{No hay disco de sistema o error de disco}; esto sucede por ejemplo cuando
el disco rígido está roto y no hay otro dispositivo de booteo.
Si en cambio el BIOS encuentra un dispositivo de almacenamiento en funcionamiento,
entonces permite continuar con el procedimiento de booteo.

Ahora el BIOS lee el sector de los primeros 512 bytes del disco rígido.
Este sector se llama MBR (Master Boot Record) y normalmente contiene 2 componentes
principales: un pequeño programa de inicio específico de un sistema operativo en el comienzo,
seguido de la tabla de particiones del disco. El BIOS a esta altura ya no toma
parte de este proceso, sino que simplemente carga los contenidos del MBR en
memoria y apunta el EIP a la primera posición donde alojó esos contenidos,
para que se comience a ejecutar el programa que se debería haber traído del MBR.
