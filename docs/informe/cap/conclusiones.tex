
El trabajo produjo desafíos en varios aspectos.
El hecho de trabajar en un proyecto \textit{Open Source}
establece de por sí el obecer ciertas prácticas de desarrollo
para lograr convivir armoniosamienta con otras colaboraciones.
A la vez que era imperioso hacer primar las necesidades
de la cátedra en cuanto a la flexibilidad de los ejercicios.

Se logró aprender sobre una nueva tecnología como Docker,
la cual plantea nuevas formar de desarrollar software
a través del \textit{microservicios}.

La adaptación a las reglas establecidad por \textit{Mumukit}
fue también una tarea a destacar. La flexibilidad y la potencia
de la \textit{metaprogramación} brindada por ruby
trae como contrapartida la dificultad
en la correcta comprensión de los objetos y en el segumiento del flujo
del programa.

Este trabajo, como cierre de carrera, busca también ser un aporte
a la materia Bases de Datos, a la Universidad de Nacional de Quilmes
y a toda la comunidad de Programación.
