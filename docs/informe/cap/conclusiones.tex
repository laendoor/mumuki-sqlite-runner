
El trabajo produjo desafíos en varios aspectos.
El hecho de trabajar en un proyecto \textit{Open Source}
establece de por sí el obecer ciertas prácticas de desarrollo
para lograr convivir armoniosamienta con otras colaboraciones.
A la vez que era imperioso hacer primar las necesidades
de la cátedra en cuanto a la flexibilidad de los ejercicios.

Hubo herramientas nuevas por descubrir, principalmente
las formas desarrollo utilizando Docker y como ello
plantea una perspectiva nueva acerca de como ejecutar software.

La adaptación a las reglas establecidad por \textit{Mumukit}
fue también una tarea a destacar. La flexibilidad y la potencia
de la \textit{metaprogramación} brindada por ruby
trae como contrapartida la dificultad
en el correcto parseo y uso de los datos y el flujo recibido,
lo cual fue mitigado por la constante ayuda brindada por el equipo Mumuki.

Este trabajo pretende ser un aporte a la materia Bases de Datos,
a la Universidad de Nacional de Quilmes y a toda la comunidad
de programación, presente y futura.
\enquote{Seamos libres, lo demás no importa nada.}.


